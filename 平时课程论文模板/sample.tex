%!TEX TS-program = xelatex
%!TEX encoding = UTF-8 Unicode

%%
%% 使用 njuthesis 文档类生成南京大学本科生毕业论文的示例文档
%% 
%%

%% 
%% 南京大学本科学位论文模板

%% thesis表示毕业论文,design表示毕业设计
%% 如需Adobe字体请用
%% 如果字体不全使用Adobe选项可能会报错
%\documentclass[adobefonts, thesis]{njuthesis}
%% MacOS系统请用
%\documentclass[macfonts, thesis]{njuthesis}
%% Windows系统请用
\documentclass[winfonts, thesis]{njuthesis}
\usepackage{anyfontsize}
%% Linux系统请用
% \documentclass[linuxfonts, thesis]{njuthesis}

% %%%%%%%%%%%%%%%%%%%%%%%%%%%%%%%%%%%%%%%%%%%%%%%%%%%%%%%%%%%%%%%%%%%%%%%%%%%%%%%
% 设置论文的中文封面

% 论文标题

\title{浅析所有人都爱摸鱼定律的数学原理与应用}
\classname{摸鱼导论}
% 长标题
% \titlea{一二三四五六七八九十一二三四}
% \titleb{一二三四五六七八九十一二三}
% \titlec{一二三四五六七八九十一二三}


% 论文作者姓名
\author{小摸鱼}
% 论文作者学号
\studentid{220000000}
% 导师姓名职称
\supervisor{摸鱼老板}
% 导师职称


% % (可能存在的)第二导师姓名
% \secondsupervisor{第二导师}
% % 导师职称
% \secondsupervisortitle{导师}

% 论文作者院系
\department{摸鱼系}
% 论文作者专业方向
\major{摸鱼科学与技术}
% 论文作者的年级
\grade{2022级}
% 论文提交日期,需设置年、月、日。此属性可选,默认值为最后一次编译时的日期,精确到日。
\submitdate{2022年5月20日}

%%%%%%%%%%%%%%%%%%%%%%%%%%%%%%%%%%%%%%%%%%%%%%%%%%%%%%%%%%%%%%%%%%%%%%%%%%%%%%%
% 设置论文的英文封面

% 论文的英文标题
%\englishtitle{Thesis paper template}
% 论文作者姓名的拼音
%\englishauthor{San Zhang}
% 导师姓名职称的英文
%\englishsupervisor{Professor FengChendian}
% \englishsecondsupervisor{Professor HermitSun}
% 论文作者所在院系的英文名称
%\englishdepartment{School of Electronic Science and Engineering}
% 论文作者所在学校或机构的英文名称。此属性可选,默认值为``Nanjing University''。
%\englishinstitute{Nanjing University}
% 论文完成日期的英文形式,默认最后一次编译的时间
%\englishdate{May 20, 2018}
% 专业
%\englishinstitute{Electronic Information Science and Technology}
%%%%%%%%%%%%%%%%%%%%%%%%%%%%%%%%%%%%%%%%%%%%%%%%%%%%%%%%%%%%%%%%%%%%%%%%%%%%%%%
% 设置论文的页眉页脚
\usepackage{fancyhdr}
\definecolor{mygreen}{rgb}{0,0.6,0}
\definecolor{mygray}{rgb}{0.5,0.5,0.5}
\definecolor{mymauve}{rgb}{0.58,0,0.82}
\lstset{ %
numbers=left,   %添加代码的编号,在左侧
backgroundcolor=\color{white},   % choose the background color
basicstyle=\ttfamily,        % size of fonts used for the code
columns=fullflexible,
breaklines=true,                 % automatic line breaking only at whitespace
captionpos=b,                    % sets the caption-position to bottom
tabsize=4,
commentstyle=\color{mygreen},    % comment style
escapeinside={\%*}{*)},          % if you want to add LaTeX within your code
keywordstyle=\color{blue},       % keyword style
stringstyle=\color{mymauve}\ttfamily,     % string literal style
frame=single,
rulesepcolor=\color{red!20!green!20!blue!20},
% identifierstyle=\color{red},
language=C++,
}
%%   页眉自定义
\pagestyle{fancy}
\lhead{\includegraphics[scale=0.03]{NJU.png}}  %在此处插入NJU.png图片 图片靠左
\chead{} % 页眉中间位置内容
\rhead{\bfseries 摸鱼导论}
\renewcommand{\headrulewidth}{0.4pt}
%%%%%%%%%%%%%%%%%%%%%%%%%%%%%%%%%%%%%%%%%%%%%%%%%%%%%%%%%%%%%%%%%%%%%%%%%%%%%%%
\begin{document}
\fontsize{11pt}{10pt}\selectfont
\maketitle
% 制作英文封面
% \makeenglishtitle
% 毕业论文过程管理四页表
% \controlpage %可以将word文件交给老师签字后扫描转成pdf,然后命名为controlpage.pdf

% 论文的中文摘要


%%%%%%%%%%%%%%%%%%%%%%%%%%%%%%%%%%%%%%%%%%%%%%%%%%%%%%%%%%%%%%%%%%%%%%%%%%%%%%%
% 论文的英文摘要
%\begin{englishabstract}
%  The diversity of handwritten Chinese text make it a promising but challenging computer vision problem.
%  % 英文关键词。关键词之间用英文半角逗号隔开,末尾无符号。
%  \englishkeywords{Handwritten Chinese, Text recognition, Deep learning}
%\end{englishabstract}

%%%%%%%%%%%%%%%%%%%%%%%%%%%%%%%%%%%%%%%%%%%%%%%%%%%%%%%%%%%%%%%%%%%%%%%%%%%%%%%
% 论文的前言,应放在目录之前,中英文摘要之后
%
%\begin{preface}
%
%在过去的40年中,手写中文文本领域识别(HCTR)取得了很大的进展[1,2]。
%
%\vspace{1cm}
%\begin{flushright}
%饶安逸\\
%2018年5月15日于南大仙林
%\end{flushright}
%
%\end{preface}

%%%%%%%%%%%%%%%%%%%%%%%%%%%%%%%%%%%%%%%%%%%%%%%%%%%%%%%%%%%%%%%%%%%%%%%%%%%%%%%
% 生成论文目录
% \tableofcontents

%%%%%%%%%%%%%%%%%%%%%%%%%%%%%%%%%%%%%%%%%%%%%%%%%%%%%%%%%%%%%%%%%%%%%%%%%%%%%%%
% 生成插图清单。如无需插图清单则可注释掉下述语句。
%\listoffigures

%%%%%%%%%%%%%%%%%%%%%%%%%%%%%%%%%%%%%%%%%%%%%%%%%%%%%%%%%%%%%%%%%%%%%%%%%%%%%%%
% 生成附表清单。如无需附表清单则可注释掉下述语句。
%\listoftables

%%%%%%%%%%%%%%%%%%%%%%%%%%%%%%%%%%%%%%%%%%%%%%%%%%%%%%%%%%%%%%%%%%%%%%%%%%%%%%%
% 开始正文部分

%%%%%%%%%%%%%%%%%%%%%%%%%%%%%%%%%%%%%%%%%%%%%%%%%%%%%%%%%%%%%%%%%%%%%%%%%%%%%%%
% 学位论文的正文应以《绪论》作为第一章

\section*{摘要}
man,what can i say

关键词:摸鱼,科比,抽象

\section*{一.引言}



\section*{二、摸鱼的概念与历史}

\section*{三、所有人都爱摸鱼定律的数学原理}

\subsubsection*{1. 所有人都爱摸鱼定律的数学原理1}

\subsubsection*{2. 所有人都爱摸鱼定律的数学原理2}

\section*{四、所有人都爱摸鱼定律在数学中的应用}


% \newpage %为了将图片实例放在一起,另起一页,使用时请删掉

\section*{六、所有人都爱摸鱼定律在自然科学中的应用}

\begin{itemize}
  \item 物理学:在物理学研究中,摸鱼可以用来分析物理常数和测量数据的分布规律。例如,什么都发现不了
  \item 生物学:在生物学研究中,首位数定律可以用来分析生物数据的分布规律。例如,什么都发现不了
\end{itemize}

\section*{七、所有人都爱摸鱼定律的适用条件}

\section*{八、结论与展望 }


%\chapter{总结与讨论}

%\bibliography{sample}

%%%%%%%%%%%%%%%%%%%%%%%%%%%%%%%%%%%%%%%%%%%%%%%%%%%%%%%%%%%%%%%%%%%%%%%%%%%%%%%
% 致谢,应放在结论之后
%\begin{acknowledgement}
%\end{acknowledgement}

%%%%%%%%%%%%%%%%%%%%%%%%%%%%%%%%%%%%%%%%%%%%%%%%%%%%%%%%%%%%%%%%%%%%%%%%%%%%%%%
\end{document}